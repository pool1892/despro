\documentclass[10pt]{article}

\usepackage{style}

\title{Deskriptive Prog, Aufgabe 32, 33}

\author{Christoph Schwerdtfeger}
 
\begin{document}
\maketitle
\section{Aufgabe 32}
\subsection{LParserCore}
\begin{align*}
\zuz \mathtt{(p|||q)++>f = (p++>f)|||(q++>f)}
\\ \mathtt{(p|||q)++>f a}
\\\Longleftrightarrow \mathtt{concatMap(\lambda\ (x,y) \rightarrow\ f\ x\ y)\ (p|||q\ a)} \tag{Def. $++>$}
\\\Longleftrightarrow \mathtt{concatMap(\lambda\ (x,y) \rightarrow\ f\ x\ y)\ (p\ a\ ++\ q\ a)} \tag{Def. $|||$}
\\\Longleftrightarrow \mathtt{concat(map(\lambda\ (x,y) \rightarrow\ f\ x\ y)\ (p\ a\ ) ++\ map(\lambda\ (x,y) \rightarrow\ f\ x\ y)\ (q\ a))}\tag{Eigenschaft von map}
\\\Longleftrightarrow \mathtt{concat(map(\lambda\ (x,y) \rightarrow\ f\ x\ y)\ (p\ a\ )) ++\ concat(map(\lambda\ (x,y) \rightarrow\ f\ x\ y)\ (q\ a))} \tag{Eigenschaft von concat}
\\\Longleftrightarrow \mathtt{(p++>f a)|||(q++>f a)} \tag{Alles rückwärts}
\end{align*}
\subsection{ParserCore}
\begin{align*}
\zuz \mathtt{(p|||q)++>f \neq (p++>f)|||(q++>f)}
\end{align*}
Mittels Gegenbeispiel - 
Betrachte folgende Definition: 
\begin{verbatim}
p= many digit
q= many lower
f x y= if x == [()] then upper y else failure y
a = 'xY'
\end{verbatim}
Dann wird $\mathtt{(p|||q)++>f} $ im Gegensatz zur rechten Seite einen Fehler werfen (liegt an der Nutzung von yield in many).

\section{Aufgabe 33}
\begin{verbatim}
[x*x|x<-[1..150], x`mod`4 == 0]
[x * x | x `mod` 4 == 0,c]) [1..150])

concat (map (\x -> [x * x | x `mod` 4 == 0]) [1..150])
concat (map (\x -> if (x `mod` 4 == 0) then (x * x) else []) [1..150])

\end{verbatim}
\end{document}